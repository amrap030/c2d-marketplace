\chapter{Introduction}
\label{cha:introduction}

With the proliferation of new technologies in recent years, the amount of data increases with a potential to revolutionize essential aspects in our world. In fact, the worldwide interaction with data follows an exponentially growing trend, with a volume of 79 Zettabytes (ZB) in the last year, and an expected volume of 181 ZB in 2025 \cite{statistaDataWorldWide}. This giant pool of data can help companies to develop new business models, make better decisions through analytics and build smarter applications with machine learning models. For example, access to data can enhance the treatment of diseases in the healthcare sector \cite{koutsosAgoraPrivacyAwareData,wangBigDataAnalytics2018,suBDTFBlockchainBasedData2020} and increase productivity and efficiency in the agriculture sector \cite{elijahOverviewInternetThings2018}. More and more connected Internet of Things (IoT) devices accelerate the collection of valuable data \cite{ozyilmazIDMoBIoTData2018,lawrenzBlockchainTechnologyApproach2019}. However, without open and censorship free access to data, it cannot be used. Consequently, the economic value of data promotes the need for online data marketplaces \cite{dagevilleSnowflakeElasticData2016,krishnamachariI3IoTMarketplace2018}.

A conventional data marketplace is a two-party trading platform, where data owners, referred as sellers \emph{S}, monetize their valuable data to get compensated for data access by interested parties, referred as buyers \emph{B} \cite{banerjeeBlockchainEnabledData2019}. Such data marketplaces come in different variations. For example, there are paid subscription models for an interface to dynamic real-time data in contrast to a one-time-purchase model for access to a static resource. In addition, data marketplaces can be classified as Business-to-Business (B2B) or Business-to-Consumer (B2C) platforms. %Examples and citation or only citation?).
However, all conventional two-party data marketplaces have one thing in common -- it is not possible to guarantee fair exchange, i.e. receiving legitimate data as \emph{B} while receiving the agreed-upon payment as \emph{S}, without a Trusted Third Party (TTP) \cite{banerjeeBlockchainEnabledData2019}. Hence, the two-party model is typically extended by a centralized trading platform, where \emph{S} uploads and advertises his or her data, and the platform sells the data on behalf of \emph{S} \cite{banerjeeBlockchainEnabledData2019,daiSDTESecureBlockchainBased2020,suBDTFBlockchainBasedData2020}.

Unfortunately, centralized data trading platforms suffer from a variety of issues. A dishonest \emph{S} may be tempted to refuse data access or return manipulated data after receiving the payment, harming data availability and integrity \cite{suBDTFBlockchainBasedData2020,lawrenzBlockchainTechnologyApproach2019}. On the opposite, a dishonest \emph{B} may be tempted to never pay the price after receiving the data \cite{lawrenzBlockchainTechnologyApproach2019}. This results into a trust problem between \emph{S} and \emph{B} that a centralized trading platform should solve as a middleman. However, a malicious platform might take advantage of its monopoly to advertise products and distort rankings for their own profit \cite{ramachandranDecentralizedDataMarketplace2018}. Even worse, a malicious platform has access to advertised data, breaching privacy, \cite{banerjeeBlockchainEnabledData2019} and might resell the data without \emph{S}'s knowledge \cite{serranoPeertoPeerOwnershipPreservingData2021,suBDTFBlockchainBasedData2020,daiSDTESecureBlockchainBased2020}. All in all, the centralized platform is a Single Point of Failure (SPOF) \cite{daiSDTESecureBlockchainBased2020} that is not able to satisfy fairness, security, privacy and non-discrimination, among others \cite{banerjeeBlockchainEnabledData2019}. A decentralized infrastructure might help to reach all desirable properties \cite{ramachandranDecentralizedDataMarketplace2018}. Blockchain technology, first introduced in 2008 by Satoshi Nakamoto \cite{nakamotoBitcoinPeertoPeerElectronic}, provides a promising approach to that.

Any digital data marketplace, whether centralized or decentralized, needs some specific key components and features. This includes: (i.) \emph{Fairness}; (ii.) \emph{Transparency}, \emph{Privacy} and \emph{Security}; (iii.) \emph{Regulation}; as well as (iv.) \emph{Efficiency}, according to Banerjee and Ruj \cite{banerjeeBlockchainEnabledData2019}. Ramachandran et al. \cite{ramachandranDecentralizedDataMarketplace2018} complements this by functional requirements such as: (v.) \emph{Posting and Discovery}; (vi.) \emph{Data Transfer and Payments}; (vii.) \emph{Metadata Organization}; (viii.) \emph{Data Quality - Buyer and Seller
Ratings}; (ix.) \emph{Data Quality - Curation and Recommendations}; and (x.) \emph{Identity - and Access Control Management (IAM)}. These requirements are surrounded by the buyer and seller, as depicted in Figure \ref{fig:components}.

\begin{figure}[!htb]
    \centering
    \includegraphics[width=13cm]{images/components.png}
    \caption[Key components and features of a digital data trading platform]{Key components and features of a digital data trading platform. All components in red show the aspired enhancements by the contributions of this thesis.}
    \label{fig:components}
\end{figure}

"Blockchain-based application architectures benefit from a set of unique properties including immutability and transparency of cryptographically-secured and peer-recorded transactions, which have been agreed upon by network consensus" \cite{eberhardtBlockchainInsightsOffChaining2017}. According to that, it is trivial to implement sufficient \emph{Transparency} and \emph{Security} in blockchain-based data marketplaces. Furthermore, it turns out the non-trivial \emph{Fairness} problem in a two-party trading relationship seems to be solved by Blockchain technology in one atomic swap \cite{dziembowskiFairSwapHowFairly2018,liZKCPlusOptimizedFairexchange2021}. However, \emph{Privacy}, \emph{Regulation} and \emph{Efficiency} remains a bigger problem for blockchain-based data marketplaces due to limitations of public permission-less Blockchains such as Bitcoin \cite{nakamotoBitcoinPeertoPeerElectronic} and Ethereum \cite{buterinNEXTGENERATIONSMART}.

Public permission-less Blockchains are peer-to-peer networks which inherently validate and process transaction data at every node, in order to guarantee network consensus. By virtue of its design, all data is available at each node that makes this system purposely public in favour of transparency properties. However, data sharing often incorporates Personal Identifiable Information (PII) and confidential data. Consequently, storing and computing private data on public permission-less Blockchains is conflicting with Privacy requirements, enforced by regulations such as the European General Data Protection Regulation (GDPR) \cite{european_commission_regulation_2016}. Nevertheless, in any case it is not advisable to store large amounts of data in Blockchains due to block size limitations, Blockchain bloating and extremely high associated storage costs.

Off-chain storage solutions are suggested to overcome this limitation. In particular the Content-Addressable Storage Pattern \cite{eberhardtBlockchainInsightsOffChaining2017} seems to be a reasonable solution to store large amounts of data while keeping key properties of Blockchains such as immutability. Decentralized peer-to-peer networks such as IPFS \cite{benetIPFSContentAddressed2014}, Filecoin \cite{filecoin} and SWARM \cite{swarm} provide a promising approach to that. However, these networks suffer from the same Privacy issues as Blockchains due to open access and data replication across the network.

Another problem seems to be the enforcement of access and usage regulations in a distrusted setting without a TTP. A seller loses full control over his data when the buyer receives it. Hence, he or she is not able to enforce geographical usage restrictions or purpose limitation for example. This especially violates the principle of purpose limitation codified in Art. 5 of the GDPR \cite[Art. 5 (1 b)]{european_commission_regulation_2016}. Given a malicious buyer, he or she might even resell the purchased dataset without the seller's knowledge.

\noindent I observe, that most of the problems occur when data leaves the boundaries of the seller, i.e. a raw dataset is moved to a storage platform and/or moved to the buyer, where it subsequently is processed. While storage obstacles might be addressed by data encryption, \emph{Regulation} and \emph{Privacy} still remains a problem, when the buyer decrypts the raw dataset. According to that, I suggest to flip the strategy, i.e. raw datasets never leave the boundaries of the seller. Hence, the seller only moves the computed result of statistical queries and aggregations to the buyer, thereby protecting PII and confidential data. This paradigm is referred as \emph{Compute-to-Data}. However, a problem of this paradigm is that, given a malicious seller, the buyer cannot verify if the result has been computed correctly. This leads to the following research questions:
%Verifiable Off-chain Computation (VOC) has been suggested to address this problem \cite{eberhardtOffchainingModelsApproaches2018,eberhardtBlockchainInsightsOffChaining2017}.

%This thesis focuses on \emph{Privacy}, \emph{Fairness} and \emph{Regulation} enhancements in blockchain-based data trading platforms, by combining Compute-to-Data with Verifiable Off-chain Computation (VOC). It implicitly targets the \emph{Data Transfer} and \emph{Payment} process as well as \emph{IAM} -- some of the fundamental functional requirements, as depicted in Figure \ref{fig:components}.

%weitere problems je nach art des marketplaces
\begin{enumerate}
    \item What components are mandatory for a blockchain-based data trading platform and how to apply the Compute-to-Data paradigm on them?
    \item How can the Compute-to-Data paradigm be made verifiable without sacrificing for privacy?
    \item How practical is the proposed solution?
\end{enumerate}