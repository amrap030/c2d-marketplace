\chapter{Related Work}
\label{cha:relatedwork}

Blockchain-based data trading platforms have been researched for a long time. There are a variety of different approaches where Blockchains are primarily used in almost every implementation (i.) as a transparent tamper-proof log of transactions; (ii.) as an enforcement point for access control decisions; and (iii.) as a medium to ensure fair exchange. Off-chain computation is then leveraged in conjunction to enhance privacy, ownership and regulation of datasets.

In the most simple implementation in \cite{ozyilmazIDMoBIoTData2018,ramachandranDecentralizedDataMarketplace2018,banerjeeBlockchainEnabledData2019}, sellers encrypt their dataset before publishing it to the Blockchain or to a different storage system. Decryption keys are then distributed securely to the buyer after purchasing a dataset. However, when the buyer decrypts the dataset, regulation and privacy can't be guaranteed anymore.

Truong et al. rely on encrypted datasets as well, however with a more sophisticated key-distribution scheme \cite{truongSecureDecentralizedSharing2019}. According to that, prefix encryption, as a form of Hierarchical Identity-Based Encryption (HIBE), links the decryption key to an identity, and therefore simplifies key distribution and allows fine-grained access control. Nevertheless, privacy and regulation remains a problem after decrypting the dataset.

Serrano and Cuenca \cite{serranoPeertoPeerOwnershipPreservingData2021} enhance data ownership, privacy and misusage against regulations with Homomorphic Encryption (HE). According to that, the buyer can perform arbitrary computations on encrypted datasets, while the result is decrypted by the seller. Hence, the dataset never leaves the boundaries of the seller. %has output verifiability?

A different approach is followed by Enigma \cite{shrobeEnigmaDecentralizedComputation2018} which uses Secure Multi-Party Computation (sMPC or MPC) for data queries to build a platform with guaranteed privacy by design. With MPC, every party in the protocol has only access to a meaningless piece of data, whereas only the buyer finally receives the result of the computation.

Dai et al. proposes a secure data trading ecosystem (SDTE) \cite{daiSDTESecureBlockchainBased2020} where buyers pay for the analysis of the seller’s dataset in the form of statistical queries and aggregations. The processing is protected in a Trusted Execution Environment (TEE), specifically using Intel’s Software Guard Extension (SGX) enclaves. %\footnote{TODO}.
Hynes et al. \cite{hynesDemonstrationSterlingPrivacypreserving2018a} as well as Xiao et al. \cite{xiaoPrivacyGuardEnforcingPrivate2020} follow similar approaches, while the latter adds (i.) a novel way to commit the computation result onto the Blockchain; and (ii.) a verifiable proof to show compliance to data usage policies as a consumer. However, TEE's unfortunately have been shown to be susceptible to side-channel attacks \cite{brasserSoftwareGrandExposure,biondoGuardDilemmaEfficient}.

The most privacy preserving approach is followed in \cite{fotiouPrivacypreservingStatisticsMarketplace2021} by Fotiou et al. with a computation result based on Differential Privacy (DP). "DP addresses the paradox of learning nothing about an individual, while learning useful information about a population" \cite{tsaloliDifferentialPrivacyMeets}. This implicitly introduces an error margin into the computation result. However, this paper does not implement output verifiability. Albeit, \cite{narayanVerifiableDifferentialPrivacy2015,tsaloliDifferentialPrivacyMeets,koutsosAgoraPrivacyAwareData} show that output verifiability with DP is possible by the use of Zero-knowledge proofs (ZKP).