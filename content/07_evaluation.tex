\chapter{Evaluation}
\label{cha:evaluation}

The outcome of this thesis will be evaluated according to predetermined functional and non-functional requirements. As already described in the previous sections, this thesis aims to enhance \emph{Privacy}, \emph{Fairness} and \emph{Regulation} in blockchain-based data trading platforms by combining Compute-to-Data and Verifiable Off-chain Computation. These 3 requirements are hard to measure numerically, but the evaluation will include an in-depth discussion about the pros and cons of the proposed system design accordingly, and compared to related work. %However, if I manage to implement verifiable Differential Privacy, privacy gets indeed measurable. % Possibly in form of Threat Model?

The second part of the evaluation will include multiple benchmarks with regards to the practicality, also referred as efficiency, of the proposed system design. Interestingly, \emph{Efficiency} is one of the key features of any digital data trading platform as depicted in chapter \ref{chapter:problem}. %Hence, the result of this evaluation will provide a decent proposition about the real 
Therefore, I will analyze efficiency of on-chain and off-chain components individually, according to costs, scalability and computation time.

One of the most suitable measurements for on-chain components are the costs for each transaction, with regards to gas fees. This is important because the proposed system design is only feasible as a real-world system when costs are reasonably low for the buyer and seller. Since the Blockchain is primarily used to secure the protocol, costs can be compared to conventional buyer and seller protection systems from Ebay or PayPal for example. The most important on-chain transaction will probably be the verification of the zero-knowledge proof, which has to be cheaper than the on-chain execution in the first place. Benchmarks will vary in the size of the dataset and the computation algorithm.

The second analysis includes the evaluation of off-chain components. According to that, the ZKP, which is composed of different phases, is the most important evaluation. I will benchmark especially the time for the one-time setup as well as the proof generation time. The proof generation time will probably have the most impact on the practicality of the proposed system design. Benchmarks will again vary in the size of the dataset and the computation algorithm. Furthermore, I will take the underlying hardware with regards to the computational power into account for this analysis. 

The entire system can be measured in the overall time and amount of exchanged messages until the trade between buyer and seller is completely fulfilled. All benchmarks can be compared to conventional data marketplaces as well as related work, and used to construct suggestions to improve the system design in future work.

THREAT MODEL?